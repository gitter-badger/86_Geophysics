\documentclass[12pt]{article}
\usepackage{pmmeta}
\pmcanonicalname{MercatorProjection}
\pmcreated{2013-03-22 15:19:53}
\pmmodified{2013-03-22 15:19:53}
\pmowner{acastaldo}{8031}
\pmmodifier{acastaldo}{8031}
\pmtitle{Mercator projection}
\pmrecord{5}{37145}
\pmprivacy{1}
\pmauthor{acastaldo}{8031}
\pmtype{Definition}
\pmcomment{trigger rebuild}
\pmclassification{msc}{86A30}
\pmrelated{RiemannSphere}
\pmrelated{ConformalityOfStereographicProjection}
\pmrelated{InverseGudermannianFunction}

\endmetadata

% this is the default PlanetMath preamble.  as your knowledge
% of TeX increases, you will probably want to edit this, but
% it should be fine as is for beginners.

% almost certainly you want these
\usepackage{amssymb}
\usepackage{amsmath}
\usepackage{amsfonts}

% used for TeXing text within eps files
%\usepackage{psfrag}
% need this for including graphics (\includegraphics)
%\usepackage{graphicx}
% for neatly defining theorems and propositions
%\usepackage{amsthm}
% making logically defined graphics
%%%\usepackage{xypic}

% there are many more packages, add them here as you need them

% define commands here
\begin{document}
In a Mercator Projection the point on the sphere (of radius R) with longitude $L$ (positive East) and latitude $\lambda$ (positive North) is mapped to the point in the plane with coordinates $x,y$:

$$
x = R L
$$
$$
y = R \ln(\tan( \frac{\pi}{4} + \frac{\lambda}{2}))
$$

The Mercator projection satisfies two important properties: it is conformal, that is it preserves angles, and it maps the sphere's parallels into straight line segments of length $2\pi R$.  (A parallel of latitude means a small circle comprised of points at a specified latitude).

Starting from these two properties we can derive the Mercator Projection.  First note that a parallel of latitude $\lambda$\ has length $2\pi R \cos( \lambda)$.    To make the projections of the parallels all the same length a stretching factor in longitude of $\frac{1}{\cos( \lambda)}$ will have to be applied.  For the mapping to be conformal, the same stretching factor must be applied in latitude also.  Note that the stretching factor varies with $\lambda$ so to map a specified latitude $\lambda_0$ to an ordinate $y$ we must evaluate an integral.
$$
y = \int_{0}^{\lambda_0} (1/\cos( \lambda)) d\lambda
$$
Early mapmakers such as Mercator evaluated this integral numerically to produce what is called a Table of Meridional Parts that can be used to map $\lambda_0$ into y.  Later it was noticed that the integral of one over cosine actually has a closed form, leading to the expression for $y$ shown above.
%%%%%
%%%%%
\end{document}
