\documentclass[12pt]{article}
\usepackage{pmmeta}
\pmcanonicalname{ConformalityOfStereographicProjection}
\pmcreated{2013-03-22 18:47:44}
\pmmodified{2013-03-22 18:47:44}
\pmowner{pahio}{2872}
\pmmodifier{pahio}{2872}
\pmtitle{conformality of stereographic projection}
\pmrecord{8}{41594}
\pmprivacy{1}
\pmauthor{pahio}{2872}
\pmtype{Derivation}
\pmcomment{trigger rebuild}
\pmclassification{msc}{86A30}
\pmclassification{msc}{51M04}
\pmclassification{msc}{32C15}
\pmrelated{IntersectionOfSphereAndPlane}
\pmrelated{MercatorProjection}

% this is the default PlanetMath preamble.  as your knowledge
% of TeX increases, you will probably want to edit this, but
% it should be fine as is for beginners.

% almost certainly you want these
\usepackage{amssymb}
\usepackage{amsmath}
\usepackage{amsfonts}
\usepackage{amsthm}

\usepackage{mathrsfs}
\usepackage{pstricks}
\usepackage{pst-plot}

% used for TeXing text within eps files
%\usepackage{psfrag}
% need this for including graphics (\includegraphics)
%\usepackage{graphicx}
% for neatly defining theorems and propositions
%
% making logically defined graphics
%%%\usepackage{xypic}

% there are many more packages, add them here as you need them

% define commands here

\newcommand{\sR}[0]{\mathbb{R}}
\newcommand{\sC}[0]{\mathbb{C}}
\newcommand{\sN}[0]{\mathbb{N}}
\newcommand{\sZ}[0]{\mathbb{Z}}

 \usepackage{bbm}
 \newcommand{\Z}{\mathbbmss{Z}}
 \newcommand{\C}{\mathbbmss{C}}
 \newcommand{\F}{\mathbbmss{F}}
 \newcommand{\R}{\mathbbmss{R}}
 \newcommand{\Q}{\mathbbmss{Q}}
\newcommand*{\norm}[1]{\lVert #1 \rVert}
\newcommand*{\abs}[1]{| #1 |}

\newtheorem{thm}{Theorem}
\newtheorem{defn}{Definition}
\newtheorem{prop}{Proposition}
\newtheorem{lemma}{Lemma}
\newtheorem{cor}{Corollary}
\begin{document}
\begin{center}
\begin{pspicture}(-6,-4)(7.5,3)
\psdot(0,0)
\psdot[linecolor=blue](0,1.98)
\rput(0,2.3){$N$}
\rput(4.3,-3.1){$P$}
\rput(1.2,1.1){$P'$}
\rput(-0.72,-1.3){$m$}
\pscircle[linecolor=blue](0,0){2.04}
\psplot[linecolor=blue]{-0.6}{0}{3.33 0.36 x x mul sub sqrt mul}
\psplot[linecolor=blue]{-0.6}{0}{-3.33 0.36 x x mul sub sqrt mul}
\psplot[linestyle=dotted]{0}{0.6}{3.33 0.36 x x mul sub sqrt mul}
\psplot[linestyle=dotted]{0}{0.6}{-3.33 0.36 x x mul sub sqrt mul}
\psplot[linecolor=blue]{-2}{2}{-0.3 4 x x mul sub sqrt mul}
\psplot[linestyle=dotted]{-2}{2}{0.3 4 x x mul sub sqrt mul}
\psline[linewidth=0.06](-5.5,-3)(4,-3)
\psline[linewidth=0.05](4,-3)(6.5,-1)
\psline(-5.5,-3)(-3,-1)
\psline(-3,-1)(-1.73,-1)
\psline(1.73,-1)(2.3,-1)
\psline(2.5,-1)(6.5,-1)
\psline(-1.2,-3)(0,-2)
\psline[linestyle=dashed](0,2)(0.89,0.89)
\psline[linecolor=red](0.89,0.89)(4,-3)
\psdot[linecolor=blue](0.89,0.89)
\psline(4,-3)(3.5,-1)
\psline(4,-3)(4.5,-1)
\psdot[linecolor=red](4,-3)
\rput(3.8,-1.3){$l$}
\rput(4.7,-1.4){$l_1$}
\rput(-6,-4){.}
\rput(7.4,3){.}
\end{pspicture}
\end{center}

We shall see, that the stereographic projection
\begin{align}
P \;\mapsto\; P'
\end{align}
mapping the points $P$ of the closed complex plane \,$\mathbb{C}\cup{\infty}$\, \PMlinkname{bijectively}{Bijective} onto the points $P'$ of the Riemann sphere preserves the angles between two curves, i.e. it is conformal.\, Therefore, it is used as a \PMlinkescapetext{map projection}.\\

Let $l$ be a line in the complex plane.\, If it passes through the origin, then (1) maps it onto a meridian (as $m$) of the sphere.\, If $l$ don't pass through the origin, we can think all lines passing through the North Pole $N$ and a point of $l$; these lines form a plane $\pi$ which intersects the sphere along a circle $c$.\, Thus (1) maps any line onto a circle which passes through the point $N$.\, Moreover, the tangent line $t$ of the circle $c$ at $N$ is parallel to the line $l$, because it is the intersection line of the plane $\pi$ and the tangent plane of the Riemann sphere set at $N$, and the tangent plane is parallel to the complex plane.

Think then two lines $l$ and $l_1$ in the complex plane, intersecting at the point $P$, the image point of which on the Riemann sphere is $P'$.\, The image circles $c$ and $c_1$ of $l$ and $l_1$ pass through $N$ at which their tangent lines $t$ and $t_1$ are parallel to the lines $l$ and $l_1$.\, The angle formed by $c$ and $c_1$ at $N$ is thus equal to the angle between $l$ and $l_1$, similarly is the angle the circles form at the other intersecting point $P'$.\, Because obviously any two curves of the complex plane touching each other correspond two curves on the Riemann sphere also touching each other, we can infer that an angle in the plane always corresponds an equal angle on the sphere.\, Accordingly, the stereographic projection (1) is conformal.\\

\textbf{Remark.}\, If the \PMlinkname{diametre}{Diameter} of the Riemann sphere is the \PMlinkescapetext{length unit} and $z_1$ and $z_2$ are two arbitrary points of the complex plane, then the distance of the corresponding points on the Riemann sphere is
$$\frac{|z_1\!-\!z_2|}{\sqrt{(1\!+\!|z_1|^2)(1\!+\!|z_2|^2)}}.$$


%%%%%
%%%%%
\end{document}
